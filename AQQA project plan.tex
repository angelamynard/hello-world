% !TEX TS-program = pdflatex
% !TEX encoding = UTF-8 Unicode

\documentclass[11pt]{article} % use larger type; default would be 10pt
\usepackage[utf8]{inputenc} % set input encoding (not needed with XeLaTeX)

%%% Examples of Article customizationskjhuscljshbcljb\scjhs
% These packages are optional, depending whether you want the features they provide.
% See the LaTeX Companion or other references for full information.

%%% PAGE DIMENSIONS
\usepackage{geometry} % to change the page dimensions
\geometry{a4paper} % or letterpaper (US) or a5paper or....
% \geometry{margin=2in} % for example, change the margins to 2 inches all round
% \geometry{landscape} % set up the page for landscape
%   read geometry.pdf for detailed page layout information

\usepackage{graphicx} % support the \includegraphics command and options
% \usepackage[parfill]{parskip} % Activate to begin paragraphs with an empty line rather than an indent

%%% PACKAGES
%\usepackage{booktabs} % for much better looking tables
\usepackage{array} % for better arrays (eg matrices) in maths
%\usepackage{paralist} % very flexible & customisable lists (eg. enumerate/itemize, etc.)
\usepackage{verbatim} % adds environment for commenting out blocks of text & for better verbatim
%\usepackage{subfig} % make it possible to include more than one captioned figure/table in a single float
\usepackage[pdftex,colorlinks,linkcolor=blue]{hyperref}
\usepackage{longtable} % to split tables over 1 page where necessary.

%%% HEADERS & FOOTERS
%\usepackage{fancyhdr} % This should be set AFTER setting up the page geometry
%\pagestyle{fancy} % options: empty , plain , fancy
%\renewcommand{\headrulewidth}{0pt} % customise the layout...
%\lhead{}\chead{}\rhead{}
%\lfoot{}\cfoot{\thepage}\rfoot{}
%
%%%% SECTION TITLE APPEARANCE
%%\usepackage{sectsty}
%\allsectionsfont{\sffamily\mdseries\upshape} % (See the fntguide.pdf for font help)
%% (This matches ConTeXt defaults)

%%% ToC (table of contents) APPEARANCE
\usepackage[titletoc]{appendix}
%\usepackage[nottoc,notlof,notlot]{tocbibind} % Put the bibliography in the ToC
\usepackage[titles,subfigure]{tocloft} % Alter the style of the Table of Contents
\renewcommand{\cftsecfont}{\rmfamily\mdseries\upshape}
\renewcommand{\cftsecpagefont}{\rmfamily\mdseries\upshape} % No bold!

%%% END Article customizations

\title{OBR Air Quality Quality Assurance Project Plan*}
\author{Angela Mynard}
%\date{} % Activate to display a given date or no date (if empty), otherwise the current date is printed 

\begin{document}

\maketitle

\begin{figure}[!h]
   % \caption{My caption}
    \centering
    \includegraphics[width=3cm]{figures/logo.jpg}
 %   \label{fig:label}
\end{figure}

\textcolor{red}{*AQQA. A nice acronym. A crap name. Subject to change!}\\
This document provides the who, what, where, when and how's of the preparation, measurement and delivery stages of the Air Quality programme OBR Quality Assurance project. The document is a work in progress and will evolve as the project progresses. The idea is that the relevant sections/information can simply be extracted to produce milestone, or other reporting documentation.\\

\newpage
\section{Document History}

\begin{longtable}{|c|c|c|c|p{6cm}|}
\hline
\textbf{Version} & \textbf{Distribution} & \textbf{Modified by} & \textbf{When} & \textbf{Notes} \\
\hline
Draft & Internal & A Mynard & Feb 2019 & Very. Rough. Draft. \\
\hline
\end{longtable}


\newpage
\tableofcontents

\newpage
\section{Acronyms and Abbreviations}
\begin{tabular}{ll}
AQDQA & Air Quality Data Quality Assurance project \\
OBR & Observation Based Research \\
ADAQ & Atmospheric Dispersion and Air Quality \\
MOCCA & Met Office Civil Contingency Aircraft \\
POPS & Portable Optical Particle Spectrometer \\
TAP & Tricolour Absorption Photometer \\
AQB & Air Quality Box \\
SLR & Straight and Level Run \\
AURN & Automatic Urban and Rural Network \\
\end{tabular}
\newpage
%--------------------------------------------------------------------------------------------------------------------------------------
%--------------------------------------------------------------------------------------------------------------------------------------
\section{Project Overview}

While aerosol and gaseous pollutants in the UK are generally well-observed at the surface, and column-averaged information is increasingly available from satellite observations, there remains limited data on the vertical distribution of key pollutants in the UK boundary layer. This information is needed to support air quality model development and evaluation. This project aims to address this gap by collecting regular airborne measurements over the Southern UK for a period of 1-2 years using the Met Office Civil Contingencies Aircraft (MOCCA). In addition to delivering a long-term dataset for community use, these observations will be used to inform longstanding questions surrounding the use of high resolution in-situ observations for evaluation of regional-scale air quality models. \footnote{Taken from MOCCA Measurement Campaign Project Overview.doc and  “A database of UK pollutant vertical distributions to support air quality model development and evaluation” 10th July 2018, Justin Langridge}

Key pollutants for observation have been defined as PM\textsubscript{2.5}, PM\textsubscript{10}, NO\textsubscript{x} and O\textsubscript{3} by the Atmospheric Dispersion and Air Quality (ADAQ) group. These pollutants will be measured using the below instruments, specified by the OBR MOCCA team.

\begin{itemize}
\item  NO\textsubscript{2} CAPS ( ), 
\item a Portable Optical Particle Spectrometer (POPS) to measure fine mode aerosol 
\item a Tricolour Absorption Photometer (TAP) to measure ambient black carbon concentrations and light absorption.  Air Quality Box (sample from Brechtel isokinetic inlet)
\item Nephelometer (sample from Brechtel isokinetic inlet)
\end{itemize}

Observations will take place during one MOCCA flight per week, over the duration of the project. The MOCCA Air Quality instrument suite and resulting data will undergo routine quality assurance assessment before data is made available to the ADAQ (or other departments) for analysis. This document provides details, methods and processes required to achieve these analysis.  

Following QA analysis, the data will be made available to \textcolor{red}{customers/researchers within XX hours/days/MONTHS! via some OBR Database}. The resulting database will be used over the next 4 years for model development by the ADAQ group, who will do the bulk of the data analysis. There is potential to included limited data from previous measurement campaigns in the database.

Something about vertical profiles and seasonal variability....
%--------------------------------------------------------------------------------------------------------------------------------------
%--------------------------------------------------------------------------------------------------------------------------------------
\section{Deliverables}

Note this is a summary of project milestones.\footnote{Details from MOCCA Measurement Campaign Project Overview.doc}. A more detailed schedule for each sub-project can be found \textcolor{red}{in the applicable sections/or somewhere else}. 

Form JL: Meeting with Matt first would be a good idea as he will be able to advise who else to contact. I’d be happy to join for this meeting once we’re ready. 

\textbf{Preparation phase - FY 18/19}

\begin{itemize}
\item	Dec 2018: Specification of new instrumentation, installation timeline, and QA methodology (calibrations and data quality control)
\item 	Mar/April 2019: Complete installation and certification of new instrumentation.
\item 	Mar 2019: Document describing flight sortie plans, instrument calibration and QA procedures, data analysis and archiving plans. NOTE Action JK to find out if March milestone for the QA document can be put back to April. JL advised the QA document will be a "short” document that provides an overview of the systems and processes in place. Documents of greater detail (e.g. calibration processes, software architecture) will be internal to OBR. The QA document will also be used to satisfy the Board that sufficient progress has been made.
\item     April: QA framework in place.
\item 	Total time: 6 months - Joss (3 months), Andy (1 month), Dave T (1 month), Angela (1 month)
\end{itemize}

\textbf{Measurement phase FY 19/20}
\begin{itemize}
\item	April 2019: Start of flight operations.
\item 	April 2019 - ongoing: Routine QA.
\end{itemize}

\textbf{Delivery phase 20} 
\begin{itemize}
\item  Mar 2020: Summary report on 1 year dataset encompassing data availability and quality. 
\item  Mar 2020: Report on initial data analysis including case-study model comparisons and observational constraint of pollutant variability (joint with ADAQ) 
\end{itemize}
Total: 12 months (scientist time (7 months), instrument time (5 months)) 

%--------------------------------------------------------------------------------------------------------------------------------------
%--------------------------------------------------------------------------------------------------------------------------------------
\section{Project organisation}

\subsection{Contacts and roles}

Note that contact details are not provided as they are all internal contacts and therefore available via MetNet. Project Board members are highlighted in grey. \\


\begin{tabular}{|p{6cm}|p{9cm}|}
\hline
\textbf{Name} & \textbf{AQQA project role} \\ \hline

\textbf{Board Members} & \\ \hline
 & Project Executive \\ \hline
 & Senior User \\ \hline
 & Senior Supplier \\ \hline

\textbf{OBR Project Team} & \\ \hline
Angela Mynard & Project Manager, QA Software Developer \\ \hline
Justin Langride & Head of OBR \\ \hline
Rob King & OBR Manager  \\ \hline
Joss Kent & MOCCA Senior Scientist and Instrument Specialist  \\ \hline
Andy Wilson & \\ \hline
Dave Tiddeman & \\ \hline

\textbf{Customer Contacts} & \\ \hline
Matt Hort & Clean Air Project Science Leader and Head of Atmospheric Dispersion and Air Quality (ADAQ) \\ \hline
Paul Agnew & Air Quality Team Manager in ADAQ\\ \hline
Debbie O’Sullivan & Manager of a new Clean Air team within ADAQ \\ \hline
Fiona O’Connor & Science manager of atmospheric composition team in the Hadley Centre \\ \hline
\end{tabular}

\subsection{OBR Organogram}

Required? Possibly.

\subsection{Stakeholder Management and Communications Plan}
Include throughout design phase and execution phase.

%--------------------------------------------------------------------------------------------------------------------------------------
%--------------------------------------------------------------------------------------------------------------------------------------
\newpage
\section{OBR Resource plan}

%--------------------------------------------------------------------------------------------------------------------------------------
%--------------------------------------------------------------------------------------------------------------------------------------
\newpage
\section{Reporting}

\subsection{Preparation phase - FY 18/19 Initiation and definition phases}
e.g. Project status. Putting stuff in place.
To whom
Frequency 
Format

\subsection{Measurement phase} 
e.g.  Start of flight operations.
To whom
frequency 
Format

\subsection{Delivery phase} 
Mar 2020: Summary report on 1 year dataset encompassing data availability and quality. Report on initial data analysis including case-study model comparisons and observational constraint of pollutant variability (joint with ADAQ) 
To whom
frequency 
aliufhlaihcliahscks Format

\subsection{Delivery phase} 
Mar 2020: Summary report on 1 year dataset encompassing data availability and quality. Report on initial data analysis including case-study model comparisons and observational constraint of pollutant variability (joint with ADAQ) 
To whom
frequency 
Format
% something different in here
%--------------------------------------------------------------------------------------------------------------------------------------
%--------------------------------------------------------------------------------------------------------------------------------------
\newpage
\section{Storing and archiving}
Data
Instrument records
QA Code
QA code output
Documentation (Project Plans, reports, Software manual) 

% --------------------------------------------------------
\newpage
\section{Risk Management Plan}

Strategy. And link to my RAID log.

\newpage
\section{Change management}

Strategy. And link to my RAID log.

\begin{figure}[!h]
    \caption{something about change control goes here}
    \centering
    \includegraphics[width=10cm]{figures/change_control_process.png}
    \label{fig:changecontrol}
\end{figure}


\newpage
\section{Scope management}

\newpage
\section{Procurement management}
Is there any? Maybe for calibration materials/equipment.... This might fall under JK's side of the project.

\newpage
\section{Project closure and ongoing support and future plans}
Section on Future maintenance (routine processing, code maintenance).
\pagebreak

%-----------------------------------------------------------------------------------
\newpage
\section{Flight Plans}

\textcolor{red}{X} flight plans have been developed to provide vertical profiles in a range of meteorological conditions. The sortie templates can be found in appendix N.
 
The flight plans take into consideration
\begin{enumerate}
\item \textbf{Aircraft range}
Endurance 2up – approx. 4 hours
\textcolor{red}{are double flights an option? Fuel (who meets the costs, procurement method, who will organise fuel delivery?}

\item \textbf{Operating areas}
\textcolor{red}{Where do we want to go. Any restricted areas?}

\item \textbf{Speed}
Science speed – 160knts / 80m/s 
\textcolor{red}{higher speed transits possible?}

\item \textbf{Minimum/maximum operating altitudes}
Max alt-  FL250
Min Alt - 500’
\textcolor{red}{do these vary dependent on met conditions? Poss not - 500' is pretty high!}

\item \textbf{Ascent and descent rates}
Ascent rate -  800’/min
Descent rate 1000’/min 
\textcolor{red}{how do the instruments perform in profiles - conf data in SLR's only. Do the instruments suffer from fast profiles - do we need to vary speeds?}

\item \textbf{Coordination with ground-based sites}
The Automatic Urban and Rural Network (AURN) is the UK's largest automatic air quality monitoring network, used for compliance reporting against the Ambient Air Quality Directives. See \href{https://uk-air.defra.gov.uk/networks/network-info?view=aurn}{the AURN website} for more information. Parameters measured include oxides of nitrogen (NO\textsubscript{x}), sulfur dioxide (SO\textsubscript{2}), ozone (O\textsubscript{3}), carbon monoxide (CO) and particles (PM\textsubscript{10} and PM\textsubscript{2.5}). A map of AURN sites can be found here: \href{https://uk-air.defra.gov.uk/interactive-map?network=aurn}{interactive map}.

Action AM to research the AURN network of ground sites, what measurements do they make, and which ones are in range of the Bournemouth/MOCCA.
Any other sites (Chilbolton? Cardington! available?

\item \textbf{Scientific aim (customer requirements)}
Fiona O'Connel (Science manager of atmospheric composition team in the Hadley Centre) wants south UK long transit to show fluxes from the continent.
e.g. Continental fluxes? Urban fluxes? Rural/ocean backgrounds? Urban traffic?

AM to figure out where we want to go. Work with pilots/ops to see what is possible. 

\item \textbf{Met conditions}
e.g. avoid cloud. 

\item \textbf{Manouvers}
SLR's, profiles
\end{enumerate}

\subsection{Flight plan schedule}

\begin{tabular}{|p{2cm}|p{5cm}|p{3cm}|p{3cm}|p{3cm}|}
\hline
\textbf{When}					\textbf{What} 								& \textbf{Where} 		& \textbf{Who}					&\textbf{Depends on...}  \\ \hline
							  	& \textbf{Preparation phase}					&						& 								&						\\ \hline
In progress						& Gather info 								& ASAP 					& Angela Mynard 				&						\\ \hline
ASAP							& Meet with Jossticles. 						& by end of next week 	& Angela Mynard and Joss Kent	&						\\ \hline 
Mid March						& First issue sorties finished 					& end of Feb? 			& Angela Mynard 				&						\\ \hline
Mid March						& Issue to JK, JL et al for review.				& 						& Angela Mynard 				& 						\\ \hline
End March						&  Revise to V2 and re-issue for OBR approval.	& 						& Angela Mynard 				& 						\\ \hline
April							& Issue V2 (or V2.1 if applicable) to ADAQ.	&|						& Angela Mynard 				&						\\ \hline
								& \textbf{Measurement phase}				&						& 								& 						\\ \hline
								& Who is does what when setting up flight		&						&								&						\\ \hline
								& \textbf{Delivery phase}						&						& 								&						\\ \hline
								& Not applicable								&						&								&						\\ \hline
\end{tabular}
\pagebreak

%-----------------------------------------------------------------------------------
\section{MOCCA Clean Air Instrument Quality Control}

Calibration, cleaning, maintenance schedules.

NO2 CAPS
-	Has the repackaged instrument changed expected data quality, frequency etc….?
AQ Box (sample from Brechtel isokinetic inlet)
-	POPS (Portable Optical Particle Spectrometer) – fine mode aerosol.
-	TAP (Tricolour Absorption Photometer) – ambient black carbon concentrations & light absorption. 
Nephelometer (sample from Brechtel isokinetic inlet)

\begin{itemize}
\item	CAPS NO2 - rack mounted in the cabin. Some tweaks to existing kit required but nothing Joss cant handle. The calibration method is currently unclear and needs further investigation.  Existing gas titration method is problematic. NPL may be producing a cal gas.
\item	AQ Box - front hold:
\item	TAP (3 wavelength black carbon photometer). No calibration required. Filters need cleaning/changing as necessary. 
\item	POP (fine mode aerosol. Made by Handex(?)).
\item	Neph.
\item 	Certification with Cranfield quoted as £90K, reduced to £79K. JK is looking at budget and seeing what is possible (purchasing, cert etc).
\item	MOCCA annual maintenance downtime is usually July. It's been bought forward to accommodate fit out of AQ instruments, which should complete mid-April to May time.
\item	Flying to potentially start of May. 
\item	Action JK to send AM manuals and calibration processes and some sample data (raw data) where possible.
\item Action AM to liaise with JK about defining the instrument QA processes, and to see where she can help.
\end{itemize}
o	Instrument metadata (website? Sharesite? Update existing?) Needed for customer reference (as well as OBR).
- Cleaning
-	Calibrations (in-house, manufacturers)
-	Cleaning
-	Probe alignment/AOA (AOA changes with Z)


%-----------------------------------------------------------------------------------
\newpage
\section{OBR Quality Assurance Software}

\begin{itemize}
\item	Action AM to lead design and documentation of the QA software. Simple flow chart - data in, ascii out. 
\item	Some of the processing already happens and has had code written. Find it, use it. AM to arrange a meeting with JL, KS (and poss DT and JK) to talk about instrument processing and what needs to be done.
\item	QA software needs to be operational as soon as possible (ideally at the start of flying but this is unlikely). Constraints will be the limiting factor (resources). 
\item	Python - need to sign up for VDI. Select that "urgent access is required" or it wont happen.
\item	Kate uses Pycharm IDE.
\item	Documentation in LaTex?
\end{itemize}
o	What instrument data needs QC’ing?
o	Outputs required
o	Start up info with ADDQ team (see Justin’s plan) – set requirements from here.
-	Output
o	Parameters
o	Format 
o	Stand conventions (Barbara Brooks/FGAM conventions) 
o	Naming conventions
o	Methods
o	Statistics required
\subsection{	Version control} %poss move to Organisation section
Git? :-/
https://xkcd.com/1296/
Configuration library

%-----------------------------------------------------------------------------------
\newpage
\section{Genral ponderings to follow up on (draft version only!)}
\begin{itemize}
\item	Amanda at Cardington. Arrange meeting while you are still around MK.
\item  	JL advised that in approx.. 6 months time, there may be future OBR work in analyzing the data to see how variable the pollutants are as a function of scale.
\item	FGAM website.
\item	Consider data handling checks (as discussed in OBR meeting) wrt sharing personal data.
\item	Offer visit to MOCCA to customers so they know how we operate.
\end{itemize}
\pagebreak

\end{document}

